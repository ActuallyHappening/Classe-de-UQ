\begin{center}

{\Large  {\bf STAT1301 Advanced Analysis of Scientific Data}}
\medskip

{\large {\bf Semester 2, 2025, Assignment 2}}

\medskip

{\large {\bf Caleb Yates s49886350}}


\end{center}

\section*{Introduction}

Throughout the report, the following syntactical shortcuts and notation will be used.

If $\X$ is a random variable of the sample space $\Omega$, an abbreviation of set notation is as follows:
\begin{align*}
\text{Abbreviate	} & \{ d : \forall d \in \Omega \text{ and } \X(d) = x \} \\
\text{As	} & \{ \X = x \}
\end{align*}

% Additionally, when thinking in terms of sets becomes obselete,
% \begin{align*}
% \text{Abbreviate	} & \p (\{ d : \forall d \in \Omega \text{ and } \X(d) = x \}) \\
% \text{As	} & \p (\X = x)
% \end{align*}
The above abbriviation will be used with inequalities as well, e.g. $\pr{\X < x}$ or $\pr{\X > x}$.

Given some random variable $\X$, there must exist a function mapping from the sample space $\Omega$ to the domain of $\X$,
which can be at most $\mathbb{R}$. This function is (intuitively) named $\X$.
This function incidentally defines the random variable, which is the motivating reason for using its letter to represent its mapping.
The notation $\mathrm{Domain}[\X]$ will be used throughout this report to indicate the domain of the function mapping $\X$ and hence
the random variable $\X$ itself by definition.

Also, $\NormalDist(\mu, \sigma)$ indicates that $\sigma$ is $\sqrt{\Var}$, aka the standard deviation. This is opposed to the syntax of $\NormalDist(\mu, \sigma^2)$. For clarity, $\sigma = $ will always be explicitely written to avoid ambiguity.

Various probability (and set) theorems are used throughout this report. For clarity, the following are named:

\begin{equation} \label{th:pr-opposites}
\pr{\X < x} = \pr{x > \X} \forall x
\end{equation}
\begin{equation} \label{th:pr-inverse}
\pr{\X < x} = 1 - \pr{\X > x} \forall x
\end{equation}
Above (\ref{th:pr-opposites}) and (\ref{th:pr-inverse}) are true for any random variable $\X$.

\begin{equation}
\label{th:normal-dist-symmetry}
\begin{aligned}
	\X \sim &\NormalDist (\mu = 0, \sigma) \\
	\implies &\pr{\X < x} = \pr{\X > -x} \\
	\iff &\pr{\X < -x} = \pr{\X > x}
\end{aligned}
\end{equation}
When $\X$ is a symmetrical distribution around 0, for example the standard normal distribution $\Z$, above (\ref{th:normal-dist-symmetry}) is true.

Also, an equivalent formula for $\E (\X)$ was used:
\begin{equation}
\label{form:expected-value-general}
\E (\X) = \sum _{c \in \Omega} \X(c) \cdot \pr{c}
\end{equation}
Recall that $\X(c)$ is the function mapping for the random variable $\X$.
For simple cases of one random variable, this can be simplified to (\ref{form:expected-value-simple}) by noticing that the sample space can be partitioned into
\begin{equation}
\label{form:expected-value-simple}
\E(\X) = \sum _{x \in \Domain{\X}} x \cdot \pr{\X = x}
\end{equation}

\newpage

%---------------------------------------------------------------------
%Q1
\section*{Question 1}

To begin, lets define the sample space

\[
\Omega = \{ (a, b) \in \{ 1, 2, 3, 4, 5, 6 \} \}
\]
\[
|\Omega| = 36
\]
Notice this follows a uniform probability distribution:
% equation labelled eq:q1-uniform
\begin{equation*}
\label{eq:q1-uniform-p}
\p (A) = \frac{|A|}{|\Omega|} = \frac{|A|}{36}
\end{equation*}


% a)
\subsection*{Part a)}

Let $X$ be a random variable representing the payout of a given dice roll $(a,b)\in \Omega$:
\[
\X ((a,b) \in \Omega) = a \cdot b
\]

% By thinking through the cases, we know $|\{ \X = x_i \}|$ for all $i \in \mathrm{Z}^+$. Let the PMF of $\X$ be $f_x$, hence:

Let $f_{\X}$ be the PMF of $\X$. Note $f_{\X}(x \in \Omega) = \p (\{ \X = x \})$. By cases, the probability distribution of $\X$ can be deduced:
\newcommand{\pq}[2]{
	\[f_{\X}(#1) = \frac{#2}{36}\]
}

\begin{multicols}{3}

\pq{1}{1}
\pq{2}{2}
\pq{3}{2}
\pq{4}{3}
\pq{5}{2}
\pq{6}{4}
\pq{8}{2}
\pq{9}{1}
\pq{10}{2}
\pq{12}{4}
\pq{15}{2}
\pq{16}{1}
\pq{18}{2}
\pq{20}{2}
\pq{24}{2}
\pq{25}{1}
\pq{30}{2}
\pq{36}{1}

\end{multicols}

For all other values $x$, $f_{\X}(x) = 0$

% b)
\subsection*{Part b)}

This makes determining the expected value of $\X$ trivial:
\begin{align*}
\E (\X) &= \sum _{c \in \Omega} \X (c) \pr{c} \\
&= \sum _{x \in \mathrm{Domain}[\X]} x \p (\{ \X = x\}) \\
&= 1 \cdot f_{\X} (1) + 2 \cdot f_{\X}(2) + \cdots 30 \cdot f_{\X}(30) + 36 \cdot f_{\X}(36) \\
&= \frac{1}{36} + \frac{4}{36} + \cdots \frac{60}{36} + \frac{36}{36} \\
&= \frac{441}{36} = \frac{49}{4} = \$12.25
\end{align*}
Therefore the expected of $\X$ is \$12.25

% c)
\subsection*{Part c)}

Evaluating $\Var(\X)$ is similarly trivial

\begin{align*}
\Var (\X) &= \E [(\X - \E (X))^2] \\
&= \sum _{c \in \Omega} (\X(c) - \frac{49}{4})^2 \p (\{ c \}) \\
&= \sum _{x \in \mathrm{Domain}[\X]} (x - \frac{49}{4})^2 \p (\{ \X = x \}) \\
&= (1 - \frac{49}{4})^2 \cdot \frac{1}{36} + (2 - \frac{49}{4})^2 \cdot \frac{2}{36} + \cdots
	(30 - \frac{49}{4})^2 \cdot \frac{2}{36} + (36 - \frac{49}{4})^2 \cdot \frac{1}{36} \\
&= \frac{11515}{144} \approx 79.97 \\
\implies \sigma _{\X} &= \sqrt{\Var(\X)} = \sqrt{\frac{11515}{144}} \approx \$8.942
\end{align*}

\newpage

\section*{Question 2}

Understanding this question in terms of a sample space isn't very fruitful.
$\Omega$ is completely unspecified, we can only deduce that $|\Omega| \geq (0, 20)$, which implies it is continuous.
$\p (A) : \exists A \in \Omega$ is also completely unknown.

\subsection*{Part a)}

Let $\X$ be the continuous random variable of algae growth as measured in grams of biomass produced.
Note $\mathrm{Domain}[\X] = (0, 20)$.

Since $\X$ is a random variable, its PDF $f_{\X}$ must sum to $1$:

\begin{align*}
1 &= \int _{c \in \Omega} \p (\{ c \}) \\
&= \int _{x \in \mathrm{Domain}[X]} \p(\{ \X = x \}) \\
&= \int _0 ^{20} c(x^2 - 60x + 800) \mathrm{d}x \\
&= c[\frac{1}{3}x^3 - 30x^2 + 800x]_{x = 0}^{x=20} \\
1/c &= [\frac{1}{3}(20)^3 - 30(20)^2 + 800(20)] - [0 - 0 + 0] \\
1/c &= \frac{20000}{3} \\
c &= \frac{3}{20000}
\end{align*}

\subsection*{Part b)}

Let $F_{\X}$ be the CDF of $\X$:

\begin{align*}
F_{\X} &= \int _{-\infinity} ^{x} f_{\X}(x) \mathrm{d}x \\
&= \int _0 ^{x} c(x^2 - 60x + 800) \mathrm{d}x \\
&= c [\frac{1}{3}x^3 - 30x^2 + 800x]_{x =0}^{x = x} \\
\frac{F_{\X}}{c} &= [\frac{1}{3} x^3 - 30x^2 + 800x] - [\frac{1}{3} 0^3 - 30\cdot0^2 + 800\cdot0] \\
\implies F_{\X} &= c(\frac{1}{3} x^3 - 30x^2 + 800x) \text{ for } 0 \leq x \leq 20 \\
&= \frac{1}{20000}x^3 - \frac{9}{2000}x^2 + \frac{3}{25}x
\end{align*}

\subsection*{Part c)}

\begin{align*}
\E(\X) &= \int _{x \in \mathrm{Domain}[\X]} xf_{\X}\mathrm{d}x \\
&= \int _0 ^{20} x \cdot c(x^2 - 60x + 800)\mathrm{d}x \\
\frac{\E(\X)}{c} &= \int _0 ^{20} x^3 - 60x^2 + 800x \mathrm{d}x \\
 &= [\frac{1}{4} x^4 - 20x^3 + 400x^2]^{x=20}_{x=0} \\
 &= [\frac{1}{4}(20)^4 - 20(20)^3 + 400 (20)^2] - [0 - 0 + 0] \\
 &= 40000 - 160000 + 160000 \\
\E(\X) &= c \cdot 40000 \\
\E(\X) &= 6 \text{ grams}
\end{align*}

\subsection*{Part d)}

\begin{align*}
% \[
& \p (\{ \X > 10 \} | \{ \X > 2 \}) \\
&= \frac{\p(\{ \X > 10 \} \cap  \{ \X > 2 \})}{\{ \X > 2 \}} \\
&= \frac{\p(\{ \X > 10 \})}{\p (\{ \X > 2 \})}
% \]
\end{align*}

From the CDF definition of $\X$, $\p (\{ \X < x \}) = f_{\X}(x)$

\begin{align*}
\implies \p(\{ \X > 10 \}) &= 1 - \p (\{ \X < 10 \}) \\
&= 1 - F_{\X}(10) \\
&= 1 - \frac{4}{5} \\
&= \frac{1}{5}
\end{align*}

\begin{align*}
\implies \p(\{ \X > 2 \}) &= 1 - \p (\{ \X < 2 \}) \\
&= 1 - F_{\X}(2) \\
&= 1 - \frac{139}{625} \\
&= \frac{486}{625}
\end{align*}

% \[
\begin{equation}
\label{eq:q2d-result}
\implies \frac{\p(\{ \X > 10 \})}{\p (\{ \X > 2 \})}
= \frac{\frac{1}{5}}{\frac{486}{625}}
= \frac{125}{486}
\approx 0.2572
\end{equation}
% \]

Therefore, the probability that the biomass exceeds 10 grams, given that it is detectable, is above in (\ref{eq:q2d-result}) $ = \frac{125}{486}$.

\newpage

\section*{Question 3}

Assume that $p = 0.25$ for all the products, not just the 25 that were sampled.

The sample space for this is again completely unspecified, and the $\p$ probability function is practically useless for this question.
For convenience, the sample space $\Omega$ is therefore defined as the domain of $\X$, representing the number of products passing the specific inspection.

\[
\Omega = \{ 1, 2, 3 ... 24, 25 \}
\]

This makes the definition of $\X$ trivial, and its domain incidentally the entire sample space:

\[
\X(a \in \Omega) = a
\]

These definition are not necessary to solve this question, however, and are included only for completeness.

% It should be noted here that $\X'$ denotes the inverse function of $\X$, which is trivially $\X$, so $\X' = \X$.

\subsection*{Part a)}

Since each product has a $p = 0.25$ probability of passing inspection, and there are 25 products, and it is assumed each inspection and product is independant of each other, $\X$ is a binomial distribution:

\[
\X \sim \mathrm{Bin}(n = 25, p = 0.25)
\]

Notes the following theorems about binomial distributions and $\X$:

\[
\p(\{ \X = x \}) = {n \choose x}p^x (1-p)^{n - x} = {25 \choose x}0.25^x \cdot 0.75^{25 - x}
\]
\[
\E(\X) = np = \frac{25}{4}
\]
\[
\Var(\X) = np(1-p) = \frac{75}{16}
\]

\subsection*{Part b)}

Let $\X_2$ be the random variable representing the probability distribution of $\X$ with an $n$ parameter
such that the probability of finding a defect-free product exceeds 99\%:


\[
\X_2 \sim \mathrm{Bin}(n, p = 0.25)
\]

\begin{align*}
\p(\{\X_2 \geq 1\}) &> 0.99 \\
0.99 &< \p(\{ \X_2 \geq 1 \}) \\
0.99 &< 1 - \p(\{ \X_2 = 0 \}) \\
0.99 &< 1 - {n \choose 0}(0.25)^0(0.75)^n \\
0.99 - 1 &< - {n \choose 0}(0.25)^0(0.75)^n \\
0.01 &> 1 \cdot 1 \cdot 0.75^n \\
\log _{0.75} 0.01 &< n \\
\implies n &> \log _{0.75} 0.01 \approx 16.008
\end{align*}

Therefore the minimum (integer) sample size is $n = 17$.

\subsection*{Part c)}

The random variable $\Y$ is dependant on $\X$.
Given a possibility $a \in \Omega$ from the sample space, $\Y(a)$ explicitely depends upon $\X(a)$ such that it exactly equals:

\begin{align*}
\Y(a \in \Omega) &= 3\X(a) - (25 - \X(a)) \\
&= 4\X(a) - 25
\end{align*}

This allows us to calculate $\E(\X)$ and $\Var(\X)$ relatively easily using probability theorems:

\begin{align*}
\E(\Y) &= \E(4\X - 25) \\
&= 4\E(\X) - 25 \\
&= 4 \cdot \frac{25}{4} - 25 \\
&= 0
\end{align*}

\begin{align*}
\Var(\Y) &= \Var(4\X - 25) \\
&= 4^2 \Var(\X) \\
&= 16 \cdot \frac{75}{16} \\
&= 75
\end{align*}

\subsection*{Part d)}

Since $\Y$ is defined in terms of $\X$, this isn't too difficult to evaluate:

\begin{align*}
\p(\{ \Y \geq 27 \}) &= \p(\{ 4\X - 25 \geq 27 \}) \\
&= \p(\{ 4\X \geq 52\}) \\
&= \p(\{ \X \geq 13 \}) \\
&\approx 0.00337
\end{align*}

This can be calculated by running 1 - pbinom(12, 25, 0.25) in R

\newpage

\section*{Question 4}

Let $\Omega = (-\infinity, +\infinity)$ in units $\dC$, representing the continuous range of possible temperatures in the storeroom.
An argument could be made to limit this to $(-\infinity, 8)$.

Let $\X$ be a random variable for the temperature inside the storeroom.

\subsection*{Part a)}

\[
\X \sim \NormalDist (\mu = 7.5\dC, \sigma = 0.3\dC)
\]

\begin{align*}
\pr{7.2 < \X < 8} &= \pr{\frac{7.2 - 7.5}{0.3} < \frac{\X - \mu}{\sigma} < \frac{8 - 7.5}{0.3}} \\
&= \pr{-1 < \Z < \frac{5}{3}} \\
&= \pr{\Z < \frac{5}{3}} - \pr{-1 < \Z} \\
&= \pr{\Z < \frac{5}{3}} - \pr{\Z > -1}
\text{	from (\ref{th:pr-opposites})} \\
&= \pr{\Z < \frac{5}{3}} - (1 - \pr{\Z < 1})
\text{	from (\ref{th:normal-dist-symmetry})} \\
&= \pr{\Z < \frac{5}{3}} + \pr{\Z < 1} - 1
\end{align*}
Using stats tables this equals $0.9525 + 0.8413 - 1 = 0.7938$.
Using R running pnorm($\frac{5}{3}$) - pnorm($-1$) = 0.7935544 $\approx$ 0.7936.

\subsection*{Part b)}

\[
\X \sim \mathit{N}(\mu, \sigma = 0.3 \dC)
\]

\begin{align*}
\p(\{ \X > 8\dC \}) &= 1\% \\
0.01 &= \p(\{ \X > 8 \}) \\
&= 1 - \p(\{ \X < 8 \})
\text{	from (\ref{th:pr-inverse})} \\
0.99 &= \p(\{ \X < 8 \}) \\
&= \p(\{ \frac{\X - \mu}{\sigma} = \frac{8 - \mu}{\sigma}\}) \\
0.99 &= \p(\{ \Z = \frac{8 - \mu}{0.3}\}) \\
\end{align*}

Let $z$ be the value which satisfies $\p(\{ \Z < z \}) = 0.99$.

\begin{align*}
\implies \frac{8 - \mu}{0.3} &= z \\
8 - \mu &= 0.3z \\
-\mu &= 0.3z - 8 \\
\mu &= 8 - 0.3z
\end{align*}

Using the stats table, $z \approx 2.33$ which implies $\mu \approx 8 - 0.3 \cdot 2.33 = 7.301\dC$.
Using R, $z = \text{qnorm}(0.99) \approx 2.326348$, which implies $\mu \approx 8 - 0.3 \cdot 2.36348 = 7.302096 \approx 7.302\dC$.

\subsection*{Part c)}

We are given no information about the parameters of $\X$
\[
\X \sim \NormalDist (\mu, \sigma)
\]

\begin{align*}
\pr{\mu - 1\dC < \X < \mu + 1\dC} &= 95\% \\
0.95 &= \pr{\mu - 1 < \X < \mu + 1} \\
0.95 &= \pr{\frac{(\mu - 1) - \mu}{\sigma} < \frac{\X - \mu}{\sigma} < \frac{(\mu + 1) - \mu}{\sigma}} \\
&= \pr{\frac{-1}{\sigma} < \Z < \frac{+1}{\sigma}} \\
&= \p(\{ \frac{-1}{\sigma} < \Z \} \cap \{\Z < \frac{+1}{\sigma} \}) \\
\end{align*}

Note that the complement of $\{ \frac{-1}{\sigma} < \Z \} \cap \{\Z < \frac{+1}{\sigma} \}$ is
$\{ \Z < \frac{-1}{\sigma} \} \cup \{ \Z > \frac{+1}{\sigma} \}$

\begin{align*}
0.95 &= 1 - (\pr{\Z < \frac{-1}{\sigma} \} \cup \{ \Z > \frac{+1}{\sigma}}) \\
&= 1 - (\pr{\Z < \frac{-1}{\sigma}} + \pr{\Z > \frac{+1}{\sigma}}) \\
&= 1 - 2\pr{\Z < \frac{-1}{\sigma}} \\
0.05 &= 2\pr{\Z < \frac{-1}{\sigma}} \\
0.025 &= \pr{\Z < \frac{-1}{\sigma}} \\
1 - 0.025 &= 1- \pr{\Z < \frac{-1}{\sigma}} \\
0.975 &= \pr{\Z < \frac{+1}{\sigma}}
\text{	from (\ref{th:normal-dist-symmetry})}
\\
\end{align*}

Let $z$ be the solution to $0.975 = \pr{\Z < z}$

\begin{align*}
\implies z &= \frac{+1}{\sigma} \\
\implies \sigma &= \frac{1}{z}
\end{align*}

Using the stats table, $z \approx 1.96$ which implies $\sigma \approx \frac{1}{1.96} \approx 0.510204 \approx 0.51\dC$.
Using R $z = \text{qnorm}(0.975) \approx 1.959964$ which implies $\sigma \approx \frac{1}{1.959964} \approx 0.5102135 \approx 0.51\dC$.
