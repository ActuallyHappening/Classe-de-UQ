\begin{center}

{\Large  {\bf STAT1301 Advanced Analysis of Scientific Data}}
\medskip

{\large {\bf Semester 2, 2025}}

\medskip

{\large {\bf Assignment 2}}


\end{center}

If $\X$ is a random variable of the sample space $\Omega$, an abbreviation of set notation is as follows:
\begin{align}
\text{Abbreviate	} & \{ d : \forall d \in \Omega \text{ and } \X(d) = x \} \\
\text{As	} & \{ \X = x \}
\end{align}

Additionally, when thinking in terms of sets becomes obselete,
\begin{align}
\text{Abbreviate	} & \p (\{ d : \forall d \in \Omega \text{ and } \X(d) = x \}) \\
\text{As	} & \p (\X = x)
\end{align}
This abbriviation will be used with inequalities as well

\begin{enumerate}
%---------------------------------------------------------------------
%Q1

\item Question 1

To begin, lets define the sample space

\[
\Omega = \{ (a, b) \in \{ 1, 2, 3, 4, 5, 6 \} \}
\]
\[
|\Omega| = 36
\]
Notice this is uniform, and hence that $a$ and $b$ are independant
% equation labelled eq:q1-uniform
\begin{equation}
\label{eq:q1-uniform-p}
\p (A) = \frac{|A|}{|\Omega|} = \frac{|A|}{36}
\end{equation}

Let $X$ be a random variable representing the payout of a given dice roll $(a,b)\in \Omega$:
\[
\X ((a,b) \in \Omega) = a \cdot b
\]

% By thinking through the cases, we know $|\{ \X = x_i \}|$ for all $i \in \mathrm{Z}^+$. Let the PMF of $\X$ be $f_x$, hence:

\newcommand{\pq}[2]{
	\[f_x(#1) = \frac{#2}{36}\]
}

\pq{1}{1}
\pq{2}{2}
\pq{3}{2}
\pq{4}{3}
\pq{5}{2}
\pq{6}{4}

\pq{8}{2}
\pq{9}{1}
\pq{10}{2}
\pq{12}{4}

\vspace{0.5cm}

%---------------------------------------------------------------------
%Q2
\item A laboratory is studying the growth of algae in a controlled environment. The growth of algae is measured by the amount of biomass produced (in grams), which can be modelled by a random variable $X$ with probability density function (pdf)
\[
		f_X(x) = c \left(x^2 - 60x + 800\right)
		\quad \text{for } 0<x<20,
\]
where $c$ is a constant.


\begin{enumerate}
\item Find the value of $c$.
\hfill [2 marks]

%
\item Find the cumulative distribution function (cdf) of $X$.
\hfill [3 marks]

%
\item Find the expected value of $(X)$.
\hfill [2 marks]

%
\item The laboratory equipment can only detect biomass that exceeds 2 grams (the minimum detectable amount). What is the probability that the biomass exceeds 10 grams, given that it is detectable?
\hfill [3 marks]

\end{enumerate}

\vspace{0.5cm}

%---------------------------------------------------------------------
%Q3
\item Each day, a quality control officer inspects a random sample of 25 products from the production line of a factory. The probability of a product passing inspection (being defect-free) is 0.25. If the product passes, the factory saves \$3 in repair costs. If the product fails, the factory incurs an additional \$1 cost for re-inspection after repair.
Let $X$ be the number of products that passes inspection on a given day, and $Y$ be the net savings for the factory on that day.

\[
\sigma
\]

\begin{enumerate}
\item State the distribution of $X$, including all its parameters.
	\hfill [2 marks]

%
\item What is the minimum sample size needed so that the probability of finding at least one defect-free product exceeds 99\%?
	\hfill [3~marks]

%
\item Calculate the expected value and variance of the factory's net savings.
	\hfill [3 marks]

	%
\item What is the probability that the factory will save at least \$27 on a given day?
	\hfill [2~marks]

\end{enumerate}

\vspace{0.5cm}

%---------------------------------------------------------------------
%Q4
\item A storeroom in a warehouse maintains strict temperature control to ensure that sensitive materials are stored at optimal conditions. The temperature of the storeroom follows a normal distribution with mean $\mu$ and standard deviation $\sigma$ degrees Celsius ($^{\circ}$C). The storeroom has a temperature threshold of $8^\circ$C to avoid damaging the materials.

You may use statistical tables to answer this question, then use \textsf{R} to verify your results.

\begin{enumerate}
\item Suppose the storeroom temperature is adjusted so that $\mu = 7.5^\circ$C and $\sigma = 0.3^\circ$C. What is the probability that the temperature of the storeroom will be between $7.2^\circ$C and $8^\circ$C?
	\\\phantom{1} \hfill [3 marks]

%
\item Assume that $\sigma=0.3^\circ$C. What should $\mu$ be set to so that the storeroom temperature exceeds $8^\circ$C only 1\% of the time?
	\hfill [3 marks]

	%
\item What is the largest standard deviation $\sigma$ that will keep the temperature within $1^\circ$C of the mean with 95\% probability?
	\hfill [4 marks]

\end{enumerate}
\vspace{0.5cm}

%---------------------------------------------------------------------

\end{enumerate}
