\begin{center}

{\Large  {\bf STAT1301 Advanced Analysis of Scientific Data}}
\medskip

{\large {\bf Semester 2, 2025}}

\medskip

{\large {\bf Assignment 2}}


\end{center}

If $\X$ is a random variable of the sample space $\Omega$, an abbreviation of set notation is as follows:
\begin{align}
\text{Abbreviate	} & \{ d : \forall d \in \Omega \text{ and } \X(d) = x \} \\
\text{As	} & \{ \X = x \}
\end{align}

Additionally, when thinking in terms of sets becomes obselete,
\begin{align}
\text{Abbreviate	} & \p (\{ d : \forall d \in \Omega \text{ and } \X(d) = x \}) \\
\text{As	} & \p (\X = x)
\end{align}
This abbriviation will be used with inequalities as well

%---------------------------------------------------------------------
%Q1
\section{Question 1}

To begin, lets define the sample space

\[
\Omega = \{ (a, b) \in \{ 1, 2, 3, 4, 5, 6 \} \}
\]
\[
|\Omega| = 36
\]
Notice this is uniform, and hence that $a$ and $b$ are independant
% equation labelled eq:q1-uniform
\begin{equation}
\label{eq:q1-uniform-p}
\p (A) = \frac{|A|}{|\Omega|} = \frac{|A|}{36}
\end{equation}

\begin{enumerate}

% a)
\subsection{Part a)}

Let $X$ be a random variable representing the payout of a given dice roll $(a,b)\in \Omega$:
\[
\X ((a,b) \in \Omega) = a \cdot b
\]

% By thinking through the cases, we know $|\{ \X = x_i \}|$ for all $i \in \mathrm{Z}^+$. Let the PMF of $\X$ be $f_x$, hence:

Let $f_{\X}$ be the PMF of $\X$. Note $f_{\X}(x \in \Omega) = \p (\{ \X = x \})$. By cases, the probability distribution of $\X$ can be deduced:
\newcommand{\pq}[2]{
	\[f_{\X}(#1) = \frac{#2}{36}\]
}

\begin{multicols}{3}

\pq{1}{1}
\pq{2}{2}
\pq{3}{2}
\pq{4}{3}
\pq{5}{2}
\pq{6}{4}
\pq{8}{2}
\pq{9}{1}
\pq{10}{2}
\pq{12}{4}
\pq{15}{2}
\pq{16}{1}
\pq{18}{2}
\pq{20}{2}
\pq{24}{2}
\pq{25}{1}
\pq{30}{2}
\pq{36}{1}

\end{multicols}

For all other values $x$, $f_{\X}(x) = 0$

% b)
\subsection{Part b)}

This makes determining the expected value of $\X$ trivial:
\begin{align}
\E (\X) &= \sum _{c \in \Omega} \X (c) \p (c) \\
&= \sum _{x \in \mathrm{Domain}[\X]} x \p (\{ \X = x\}) \\
&= 1 \cdot f_{\X} (1) + 2 \cdot f_{\X}(2) + \cdots 30 \cdot f_{\X}(30) + 36 \cdot f_{\X}(36) \\
&= \frac{1}{36} + \frac{4}{36} + \cdots \frac{60}{36} + \frac{36}{36} \\
&= \frac{441}{36} = \frac{49}{4} = 12.25
\end{align}

% c)
\subsection{Part c)}

Evaluating $\Var(\X)$ is similarly trivial

\begin{align}
\Var (\X) &= \E [(\X - \E (X))^2] \\
&= \sum _{c \in \Omega} (\X(c) - \frac{49}{4})^2 \p (\{ c \}) \\
&= \sum _{x \in \mathrm{Domain}[\X]} (x - \frac{49}{4})^2 \p (\{ \X = x \}) \\
&= (1 - \frac{49}{4})^2 \cdot \frac{1}{36} + (2 - \frac{49}{4})^2 \cdot \frac{2}{36} + \cdots
	(30 - \frac{49}{4})^2 \cdot \frac{2}{36} + (36 - \frac{49}{4})^2 \cdot \frac{1}{36} \\
&= \frac{11515}{144} \approx 79.97 \\
\implies \sigma _{\X} &= \sqrt{\Var(\X)} = \sqrt{\frac{11515}{144}} \approx 8.942
\end{align}

\end{enumerate}
\vspace{0.5cm}