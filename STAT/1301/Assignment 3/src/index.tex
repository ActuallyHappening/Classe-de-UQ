\section{Question 1}

\subsection{Part a)}

Let $\Omega$ be the sample space. Therefore $\pr{\Omega} = 1$.
Adding all the joint pmf values must sum to 1:

\begin{align*}
\{ \Omega \} &= \bigcup \limits _{x} \bigcup \limits _y \{ \X = x \} \cap \{ \Y = y \} \\
\pr{\Omega} &= 1 \\
\implies 1 &= \p((\{ \X = -1 \} \cap \{ \Y = -1 \}) \cup \ldots \cup (\{ \X = 1 \} \cap \{ \Y = 1 \})) \\
&= \p(\{ \X = -1 \} \cap \{ \Y = -1 \}) + \ldots + \p(\{ \X = 1 \} \cap \{ \Y = 1 \})) \\
&= (p - \frac{1}{16}) + (\frac{1}{4} - p) + (0) +
(\frac{1}{8}) + (\frac{3}{16}) + (\frac{1}{8}) +
(p + \frac{1}{16}) + (\frac{1}{16}) + (\frac{1}{4} - p) \\
1 &= -\frac{1}{16} + \frac{4}{16} + \frac{7}{16} + \frac{1}{16} + \frac{1}{16} + \frac{4}{16} \\
1 &= 1
\end{align*}

Unfortunately, this tells us no information about $p$.
From the definition of probability, $\pr{c}$ for $c \in \Omega$ must be greater or equal to 0, $\pr{c \in \Omega} \geq 0$.
This can be used to restrict the possible values of $p$:

\begin{align*}
\p(A \subseteq \Omega) &\geq 0 \\
\implies \p(\{ \X = -1 \} \cap \{ \Y = -1 \}) &\geq 0 \\
p - \frac{1}{16} &\geq 0 \\
p &\geq \frac{1}{16} \\
\implies \p(\{ \X = 0 \} \cap \{ \Y = -1 \}) &\geq 0 \\
\frac{1}{4} - p &\geq 0 \\
p &\leq \frac{1}{4} \\
\implies \p(\{ \X = -1 \} \cap \{ \Y = 1 \}) &\geq 0 \\
p + \frac{1}{16} &\geq 0 \\
p &\leq \frac{1}{16}
\end{align*}

Therefore, $\frac{1}{16} \leq p \leq \frac{1}{4}$, and can be any value within this range.
