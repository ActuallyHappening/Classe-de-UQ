\section{Question 1}

\subsection{Part a)}

Let $\Omega$ be the sample space. Therefore $\pr{\Omega} = 1$.
Adding all the joint pmf values must sum to 1:

\begin{align*}
\{ \Omega \} &= \bigcup \limits _{x} \bigcup \limits _y \{ \X = x \} \cap \{ \Y = y \} \\
\pr{\Omega} &= 1 \\
\implies 1 &= \p((\{ \X = -1 \} \cap \{ \Y = -1 \}) \cup \ldots \cup (\{ \X = 1 \} \cap \{ \Y = 1 \})) \\
&= \p(\{ \X = -1 \} \cap \{ \Y = -1 \}) + \ldots + \p(\{ \X = 1 \} \cap \{ \Y = 1 \})) \\
&= (p - \frac{1}{16}) + (\frac{1}{4} - p) + (0) +
(\frac{1}{8}) + (\frac{3}{16}) + (\frac{1}{8}) +
(p + \frac{1}{16}) + (\frac{1}{16}) + (\frac{1}{4} - p) \\
1 &= -\frac{1}{16} + \frac{4}{16} + \frac{7}{16} + \frac{1}{16} + \frac{1}{16} + \frac{4}{16} \\
1 &= 1
\end{align*}

Unfortunately, this tells us no information about $p$.
From the definition of probability, $\pr{c}$ for $c \in \Omega$ must be greater or equal to 0, $\pr{c \in \Omega} \geq 0$.
This can be used to restrict the possible values of $p$:

\begin{align*}
\p(A \subseteq \Omega) &\geq 0 \\
\implies \p(\{ \X = -1 \} \cap \{ \Y = -1 \}) &\geq 0 \\
p - \frac{1}{16} &\geq 0 \\
p &\geq \frac{1}{16} \\
\implies \p(\{ \X = 0 \} \cap \{ \Y = -1 \}) &\geq 0 \\
\frac{1}{4} - p &\geq 0 \\
p &\leq \frac{1}{4} \\
\implies \p(\{ \X = -1 \} \cap \{ \Y = 1 \}) &\geq 0 \\
p + \frac{1}{16} &\geq 0 \\
p &\geq -\frac{1}{16}
\end{align*}
\begin{equation}
\label{th:q1a-p-domain}
\implies p \in [\frac{1}{16}, \frac{1}{4}]
\end{equation}

Therefore, $\frac{1}{16} \leq p \leq \frac{1}{4}$, and can be any value within this range.

\newpage
\subsection{Part b)}
Aim is to find $\pr{\X = \Y}$:

\begin{align*}
\pr{\X = \Y} &= \sum _{a} \p(\{ \X = a \} \cap \{ \Y = a \}) \\
&= \p(\{ \X = -1 \} \cap \{ \Y = -1 \}) + \p(\{ \X = 0 \} \cap \{ \Y = 0 \}) + \p(\{ \X = 1 \} \cap \{ \Y = 1 \}) \\
&= (p - \frac{1}{16}) + (\frac{3}{16}) + (\frac{1}{4} - p) \\
&= \frac{6}{16} = \frac{3}{8}
\end{align*}

\newpage
\subsection{Part c)}
The marginal pdf of $\X$ is $f_{\X}(x)$, which is equal to $\pr{\X = x}$ and can be manually evaluated:

\begin{align*}
\pr{\X = -1} &= \sum _y \p(\{ \X = -1 \} \cap \{ \Y = y \}) \\
&= (p - \frac{1}{16}) + (\frac{1}{8}) + (p + \frac{1}{16}) \\
&= 2p + \frac{1}{8} \\
\pr{\X = 0} &= \sum _y \p(\{ \X = 0 \} \cap \{ \Y = y \}) \\
&= (\frac{1}{4} - p) + (\frac{3}{16}) + (\frac{1}{16}) \\
&= -p + \frac{1}{2} \\
\pr{\X = 1} &= \sum _y \p(\{ \X = 1 \} \cap \{ \Y = y \}) \\
&= (0) + (\frac{1}{8}) + (\frac{1}{4} - p) \\
&= -p + \frac{3}{8} \\
\implies f_{\X}(x) = \pr{\X = x} &= \begin{cases}
2p + \frac{1}{8} & x = -1 \\
-p + \frac{1}{2} & x = 0 \\
-p + \frac{3}{8} & x = 1 \\
0 & \text{otherwise}
\end{cases}
\end{align*}

\begin{align*}
\pr{\Y = -1} &= \sum _x \p(\{ \X = x \} \cap \{ \Y = -1 \}) \\
&= (p - \frac{1}{16}) + (\frac{1}{4} - p) + (0) \\
&= \frac{3}{16} \\
\pr{\Y = 0} &= \sum _x \p(\{ \X = x \} \cap \{ \Y = 0 \}) \\
&= (\frac{1}{8}) + (\frac{3}{16}) + (\frac{1}{8}) \\
&= \frac{7}{16} \\
\pr{\Y = 1} &= \sum _x \p(\{ \X = x \} \cap \{ \Y = 1 \}) \\
&= (p + \frac{1}{16}) + (\frac{1}{16}) + (\frac{1}{4} - p) \\
&= \frac{6}{18} = \frac{3}{8} \\
\implies f_{\Y}(x) = \pr{\Y = y} &= \begin{cases}
\frac{3}{16} & y = -1 \\
\frac{7}{16} & y = 0 \\
\frac{3}{8} & y = 1 \\
0 & \text{otherwise}
\end{cases}
\end{align*}

\newpage
\subsection{Part d)}

$\X$ and $\Y$ are independent if
\begin{equation} \label{th:q1d-independence}
\p(\{ \X = x \} \cap \{ \Y = y \}) = \pr{\X = x} \cdot \pr{\Y = y} = f_\X(x) \cdot f_\Y(y)
\end{equation}
for all possible values $x$ and $y$. Therefore, this must be true for $x = -1$ and $y = 1$:

\begin{multicols}{2}
\begin{align*}
\mathrm{LHS} &= \p(\{ \X = -1 \} \cap \{ \Y = 1 \}) \\
&= p + \frac{1}{16}
\end{align*}
\begin{align*}
\mathrm{RHS} &= \pr{\X = -1}\pr{\Y = 1} \\
&= (2p + \frac{1}{8})(\frac{3}{8}) \\
&= \frac{3}{4}p + \frac{3}{64}
\end{align*}
\end{multicols}

As shown above, LHS and RHS are only equal for zero or one values of $p$. Letting LHS = RHS, we can find this exact value (or lack thereof):

\begin{align*}
p + \frac{1}{16} &= \frac{3}{4}p + \frac{3}{64} \\
\frac{1}{4}p &= \frac{3}{64} - \frac{1}{16} \\
p &= -\frac{1}{64} \cdot 4 = -\frac{1}{16}
\end{align*}

Therefore LHS = RHS only when $p = -\frac{1}{16}$, however from (\ref{th:q1a-p-domain}) this is not within the potential domain of $p$.
Therefore LHS $\neq$ RHS, showing one counterexample to (\ref{th:q1d-independence}), hence $\X$ and $\Y$ are not independent.

\subsection{Part e)}

\begin{align*}
\E(\X) &= \sum _x x\pr{\X = x} \\
&= -1(2p + \frac{1}{8}) + 0(-p + \frac{1}{2}) + 1(-p + \frac{3}{8}) \\
&= -2p - \frac{1}{8} - p + \frac{3}{8} \\
\therefore \E(\X)&= -3p + \frac{1}{4} \\
\E(\Y) &= \sum _y y \pr{\Y = y} \\
&= -1(\frac{3}{16}) + 0(\frac{7}{16}) + 1(\frac{3}{8}) \\
\therefore \E(\Y) &= -\frac{3}{16} + \frac{3}{8} = \frac{3}{16} \\
\Cov(\X, \Y) &= \E [(\X - \E(\X)(\Y - \E(\Y)] \\
\Cov(\X, \Y) &= \sum _{c \in \Omega} (\X(c) - \E(\X))(\Y(c) - \E(\Y))\pr{c} \\
&= \sum _{x,y} (x - (-3p + \frac{1}{4}))(y - (\frac{3}{16}))\p(\{ \X = x \} \cap \{ \Y = y \}) \\
\end{align*}
Expanding this sum is tedious and results in nine trinomials.
The following sum for the $\Cov(\X, \Y)$ expansion significantly reduces the algebra necessary by computing $\E(\X \Y)$ instead:

\begin{align*}
\Cov(\X, \Y) &= \E(\X \Y) - \E(\X)\E(\Y) \\
\E(\X \Y) &= \sum _{c \in \Omega} \X(c)\Y(c)\pr{c} \\
&= \sum _{x,y} xy \p(\{ \X = x \} \cap \{ \Y = y \}) \\
&= (-1)(-1)(p - \frac{1}{16}) + (-1)(0)(\frac{1}{4} - p) + (-1)(1)(0) \\
&+ (0)(-1)(\frac{1}{8}) + (0)(0)(\frac{3}{16}) + (0)(1)(\frac{1}{8}) \\
&+ (1)(-1)(p + \frac{1}{16}) + (1)(0)(\frac{1}{16}) + (1)(1)(\frac{1}{4} - p) \\
&= (p - \frac{1}{16}) - (p + \frac{1}{16}) + (\frac{1}{4} - p) \\
\therefore \E(\X \Y) &= -p + \frac{1}{8} \\
\implies \Cov(\X, \Y) &= (-p + \frac{1}{8}) - (-3p + \frac{1}{4})(\frac{3}{16}) \\
&= -p + \frac{1}{8} + \frac{9}{16}p - \frac{3}{64} \\
\therefore \Cov(\X, \Y) &= -\frac{7}{16}p - \frac{5}{64}
\end{align*}

\newpage
\section{Question 2}

$\Omega$ is continuous, which implies $\X$ is continuous and $\Y$ is continuous.
Let $D$ be the set of all $<x, y>$ that is inside (or on the boundary) of the triangle given.

\subsection{Part a)}

We are told that the joint pdf of $\X$ and $\Y$ is uniform over $D$, and assume it is $0$ everywhere else.
The area of the triangle $D$ on a cartesian plane is $A = \frac{1}{2}bh = 1$, and we know
\[
\int \limits _{d\in D} f_{\X,\Y}(d) \mathrm{d}d = 1
\]
Since $f_{\X, \Y}$ is uniform, this integral can be interpreted as the geometric volume of a triangular prism,
extruded from $D$ by $f_{\X, \Y}$
\begin{align*}
A \cdot f_{\X, \Y} = 1 \\
f_{\X, \Y} = 1
\end{align*}

Therefore the joint pdf of $(\X, \Y)$ is $f_{\X, \Y} = 1$.

\subsection{Part b)}

The set of vectors $D_2$ containing all vectors $<x, y>$ satisfying $x > y$ in $D$ forms a triangle on a cartesian plane
with vertices at $(0, 0)$ $(0.5, 0.5)$ and $(1, 0)$. The area of this triangle is exactly $\frac{1}{4}$ the
total area of $D$, since $A = \frac{1}{2}bh = \frac{1}{2} \cdot 1 \cdot 0.5 = \frac{1}{4}$. Therefore:
\begin{align*}
\pr{\X \geq \Y} &= \int \limits _{d \in D_2} f_{\X, \Y}(d)\mathrm{d}d \\
&= f_{\X, \Y} \cdot A \\
&= \frac{1}{4} \\
\therefore \pr{\X \geq \Y} &= \frac{1}{4}
\end{align*}

\newpage
\subsection{Part c)}

Since
\[
F_\X = \int _{-\infinity} ^{x} f_\X(x)\mathrm{d}x
\]
\[
f_\X(x) = \frac{\diff}{\diff x}F_\X(x)
\]

\subsubsection{PDF of X}

Figure \ref{fig:q2c-x} illustrates the geometric cases involved with evaluating $f_{\X}$:

\begin{figure}[th]
	% \centering
	\includegraphics[width=1 \textwidth]{Q2c X diagram.png}
	\caption{Geometric rendering of $D$ showing the cases of $F_\X$}
	\label{fig:q2c-x}
\end{figure}

\begin{multicols}{2}
We know
\begin{align*}
F_\X(-1) &= 0 \\
F_\X(1) &= 1
\end{align*}

\begin{align*}
F_\X(x < -1) &= 0 \\
F_\X(x > 1) &= 1
\end{align*}
\end{multicols}

Case 1: $-1 \leq x \leq 0$.
This implies $F_\X$ is the area of $\triangle\mathrm{A X_{1}^{\prime} X_1}$.
Let $\mathrm{AX_1} = x_1 = \mathrm{X_1 X'_1}$, which implies $x_1 = x + 1$:

\begin{align*}
\mathrm{Area(AX_1'X_1)} &= \frac{1}{2}bh \\
&= \frac{1}{2}x_1^2 \\
\end{align*}

Therefore
\[
F_\X(-1 \leq x \leq 0) = \frac{1}{2}(x + 1)^2
\]

Case 2: $0 < x \leq 1$.
This implies $F_\X(x)$ is the area of $\mathrm{ABX'_2X_2}$.
Note $x = MX_2$, and that $X_2X'_2 = X_2C$

\begin{align*}
\Area{\mathrm{ABX'_2X_2}} &= \Area{\triangle\mathrm{ABM}} + \Area{\mathrm{MBX_2'X_2}} \\
\Area{\triangle ABM} &= \frac{1}{2} \\
\Area{MBX'_2X_2} &= \Area{MBC} - \Area{X_2X_2'C} \\
\Area{MBC} &= \Area{ABM} = \frac{1}{2} \\
\Area{X_2X_2'C} &= \frac{1}{2}bh \\
&= \frac{1}{2} \cdot (1 - x) \cdot (1 - x) \\
&= \frac{1}{2}(1-x)^2 \\
\end{align*}

This is enough information to express $F_\X$:

\begin{align*}
\implies F_\X(0 < x \leq 1) &= (\frac{1}{2}) + (\frac{1}{2} - \frac{1}{2}(1 - x)^2) \\
&= 1 - \frac{1}{2}(1 - 2x + x^2) \\
&= 1 - \frac{1}{2} + x - \frac{1}{2}x^2 \\
&= -\frac{1}{2}x^2 + x + \frac{1}{2}
\end{align*}

Combining cases:

\begin{align*}
\implies F_\X =
\begin{cases}
0 &: x < -1 \\
\frac{1}{2}(1 + x)^2 &: -1 \leq x \leq 0 \\
-\frac{1}{2}x^2 + x + \frac{1}{2} &: 0 < x \leq 1 \\
1 &: x > 1
\end{cases}\\
\therefore f_\X(x) = \begin{cases}
x + 1 &: -1 \leq x \leq 0 \\
-x + 1 &: 0 < x \leq 1 \\
0 &: \text{otherwise}
\end{cases}
\end{align*}
This so happens to be the geometric shape of $D$ on a cartesian plane, ABC.

\subsubsection{PDF Y}
\begin{figure}[ht]
	% \centering
	\includegraphics[width=1 \textwidth]{Q2c Y diagram.png}
	\caption{Geometric rendering of $D$ showing the cases of $F_\Y$}
	\label{fig:q2c-y}
\end{figure}

\[
F_\Y(y \leq 0) = 0
\]
\[
F_\Y(y \geq 1) = 1
\]

Case $0 < y < 1$

\begin{align*}
F_\Y(y) &= \Area{AX'XC} \\
&= \Area{\triangle ABC} - \Area{\triangle XBX'} \\
&= 1 - 2 \cdot \Area{\triangle XIB} \\
\text{Note } \mathrm{IB} = 1 - y = \mathrm{IX} \\
\Area{\triangle XIB} &= \frac{1}{2}bh \\
&= \frac{1}{2}(1 - y)^2 \\
&= \frac{1}{2}(y^2 - 2y + 1) \\
\implies F_\Y &= 1 - (y^2 - 2y + 1) \\
&= -y^2 + 2y
\end{align*}

Combining cases

\begin{align*}
F_\Y(y) &= \begin{cases}
1 &: y \geq 1 \\
-y^2 +2y &: 0 < y < 1 \\
0 &: y < 0 \\
\end{cases} \\
f_\Y(y) &= \begin{cases}
-2y + 2 &: 0 < y < 1 \\
0 &: \text{otherwise}
\end{cases}
\end{align*}

The geometric interpretation of this is not as intuitive to realise.
Morph A to (0, 2), B to (0, 1) and C to (0, 0), and the initial
value of $f_\X(0) = AC$ is now placed on the y-axis.

\subsection{Part d)}

The continuous independence rule can be stated like so:

\begin{equation*}
f_{\X, \Y}(<x, y>) = f_\X(x) \cdot f_\Y(y) \quad \forall x,y\in \mathbb{R}
\end{equation*}

Suppose $<x, y> \in D$.

\begin{align*}
\text{LHS} &= f_{\X, \Y}(<x, y>) \\
&= 1 \\
\text{RHS} &= f_\Y(y) \cdot f_\X(x) \\
&= (-2y + 2) \cdot \begin{cases}
x + 1 &: -1 \leq x \leq 0 \\
-x + 1 &; 0 < x \leq 1 \\
\end{cases}
\end{align*}

Suppose further $<x, y> = <0, 0>$

\begin{align*}
\text{RHS} &= (-2 \cdot 0 + 2)(0 + 1) \\
&= 2
\end{align*}

Therefore there exists an $<x, y> \in D$ such that the independence rule fails, LHS $\neq$ RHS.
Therefore $\X$ and $\Y$ are not independent by counterexample.

\subsection{Part e)}

Due to the symmetry across the $x=0$ "line" and the fact that $\Cov(\X, \Y) = \int \limits _{x,y} xy\p(\{ \X = x\} \cap \{ \Y = y \})$
is negative (or zero) for $x < 0$ and positive (or zero) for $x > 0$, this demonstrates intuitively that $\Cov(\X, \Y) = 0$.

Somehow this seems dissatisfying to me.
$\Cov(\X, \Y) = \E(\X\Y) - \E(\X)\E(\Y)$ is another way to evaluate $\Cov(\X, \Y)$.
This method requires evaluating a double integral or equivalent.

\begin{align*}
\E(\X) &= \int _{-\infinity} ^{+\infinity} x f_\X(x) \diff x \\
&= \int _{-1} ^0 x(x+1)\diff x + \int _0 ^1 x(-x + 1) \diff x \\
&= \int _{-1} ^0 x^2+x \diff x + \int _0 ^{1} -x^2 + x \diff x \\
&= \left. \frac{1}{3}x^3 + \frac{1}{2}x^2 \right|^{x=0}_{x=-1} +
\left. -\frac{1}{3}x^3 + \frac{1}{2}x^2 \right|^{x=1}_{x=0} \\
&= (0 + 0) - (\frac{1}{3}(-1)^3 + \frac{1}{2}(-1)^2) +
(-\frac{1}{3}(1)^3 + \frac{1}{2}(1)^2) - (0 + 0) \\
&= +\frac{1}{3} + \frac{1}{2} - \frac{1}{3} + \frac{1}{2} \\
&= 1
\end{align*}

\begin{align*}
\E(\Y) &= \int_{-\infinity}^{+\infinity} yf_\Y(y) \diff y\\
&= \int _0 ^1 -2y + 2 \diff y \\
&= \left. -y^2 + 2y \right|^{y=1}_{y=0} \\
&= (-(1)^2 + 2(1)) - (0 + 0) \\
&= -1 + 2 \\
&= 1
\end{align*}


\begin{align*}
\E(\X\Y) &= \int _{-\infinity} ^{+\infinity} \int _{-\infinity} ^{+\infinity} xy f_\X(x)f_\Y(y) \diff x \diff y \\
&= \int _{-\infinity} ^{+\infinity} x\left(\int _0 ^1 y f_\Y(y)\diff y\right)f_\X(x) \diff x \\
&= \E(\Y) \int _{-1} ^{+1} x f_\X(x) \diff x \\
&= \E(\Y) \cdot \E(\X) \\
&= 1
\end{align*}

Therefore $\Cov(\X, \Y) = 1 - 1 \cdot 1 = 0$

\newpage
\section{Question 3}

The sample space $\Omega$ is parameterized by the number of orders in the month.

\[
\Omega_n = \{ (x_1, x_2, \ldots x_n) : \forall n\in \mathbb{Z} \text{ and } x_n \in \{ 1, 2, 3, 4 \} \}
\]

Let $\X_i$ be the random variable for the number of items in the i'th order.
Notice $\X_i$ are all independent and identically distributed (iid).

\[
\X_i((x_1, \ldots x_i, \ldots x_n) \in \Omega_n) = x_i
\]

From the question statement, for all $i \in \mathbb{Z}$:

\begin{align*}
\pr{\X_i = 1} &= 0.54 \\
\pr{\X_i = 2} &= 0.22 \\
\pr{\X_i = 3} &= 0.15 \\
\pr{\X_i = 4} &= 0.09
\end{align*}

Let $\T$ be the random variable for the total number of items:
\[
\T((x_1, \ldots x_n) \in \Omega_n) = \sum _{i=1} ^n x_i
\]

\subsection{Part a)}

Note $n = 500$. The question is asking to evaluate $\pr{\T < 900} \geq 0.95$, equivalently $\pr{\T > 900} < 0.05$.

\begin{align*}
\E(\X_i) &= \sum _x x \pr{\X_i = x} \\
&= 1 \cdot 0.54 + 2 \cdot 0.22 + 3 \cdot 0.15 + 4 \cdot 0.09 \\
&= 1.79 \\
\Var(\X_i) &= \E[(\X_i - \E(\X_i)^2] \\
&= \sum _{c\in \Omega} (\X_i(c) - 1.79)^2 \p(\{ c \}) \\
&= \sum _x (x - 1.79)^2 \pr{\X_i = x} \\
&= (1 - 1.79)^2\cdot 0.55 + (2 - 1.79)^2\cdot0.22  + (3 - 1.79)^2 \cdot 0.15 + (4 - 1.79)^2 \cdot 0.09 \\
&= 1.0059 \\
\E(\T) &= \sum _{i = 1} ^{500} \E(\X_i) \\
&= 500 \cdot 1.79 \\
&= 895 \\
\implies \mu_\T = 895 \\
\Var(\T) &= \sum_{i=1}^{500} \Var(\X_i) \\
&= 500 \cdot 1.0059 \\
&= 502.95 \\
\implies \sigma_\T = \sqrt{502.95} \approx 22.4265
\end{align*}

From the Central Limit Theorem (CLT), we can assume $\T$ follows a normal distribution.
$n = 500$ is a reasonably large $n$ for this to be a good approximation.

\[
\T \sim \mathcal{N}(\mu = 895, \sigma = 502.95)
\]

\begin{align*}
\pr{\T > 900} &= \pr{\frac{\T - \mu}{\sigma} > \frac{900 - 895}{22.4265}} \\
&= \pr{\Z > 0.2230} \\
&= 1 - \pr{\Z < 0.2230}
\text{Using stats table} \\
&= 1 - 0.5871 \\
&= 0.4129
\end{align*}

$\therefore$ No, there is approximately a 41\% change of exceeding the 900-item limit,
which is significantly higher than the threshold 5\%, therefore the company cannot process
$n = 500$ orders.

\subsection{Part b)}

With an unknown $n$, the aim is to find the greatest $n$ with the constraint that
$\pr{\T < 900} \geq 0.95$.
Note $\T \sim \mathcal{N}(\mu_\T, \sigma_\T)$

\begin{align*}
\E(\T) &= 1.79n = \mu_\T \\
\Var(\T) &= 1.0059n \\
\implies \sigma _\T &= \sqrt{1.0059n} \\
\pr{\T < 900} &= \pr{\frac{\T - \mu}{\sigma} < \frac{900 - \mu}{\sigma}} \\
&= \pr{\Z < \frac{900 - \mu}{\sigma}} \\
\text{let } z = \frac{900 - \mu}{\sigma} \\
\implies \pr{\Z < z} \geq 0.95 \\
\end{align*}
From stats table $z \geq 1.65$.
Ignoring the inequality until the end, and skipping algebra steps:
\begin{align*}
z = 1.65 &= \frac{900 - \mu}{\sigma} \\
1.65 \sqrt{1.0059n} &= 900 - 1.79n \\
0 &= 1.79^2n^2 + (-1.65^2 \cdot 1.0059 - 2\cdot 900)n + 900^2 \\
\end{align*}
Using the quadratic equation
\[
\implies n = 482.5 \text{ and } n = 523.96
\]
Since we know $z = \frac{900 - \mu}{\sigma} \geq 1.65$, we can reject $n = 523.96$:

\begin{align*}
1.65 &\leq \frac{900  1.79 \cdot 523.96}{\sqrt{1.0059\cdot523.96}} \\
1.65 &\leq \approx -1.577
\end{align*}

And $n = 482.5$ is reasonable, considering rounding imprecision.

\begin{align*}
1.65 &\leq \frac{900  1.79 \cdot 482.5}{\sqrt{1.0059\cdot482.5}} \\
1.65 &\leq \approx 1.649
\end{align*}

$\therefore n \approx 482.5$, which erring on the side of $\pr{\T < 900} > 0.95$ requires
us to round down to $n = 482$.
Therefore the largest number of orders the company can process in the mont is 482,
before it exceeds the limit with more than a 5\% probability.
